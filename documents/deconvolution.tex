%% This document is part of the HerschelDustMap project.
%% All content is Copyright 2015 the authors.

\documentclass[12pt, preprint]{aastex}

\newcommand{\project}[1]{\textsl{#1}}
\newcommand{\TheTractor}{\project{The~Tractor}}
\newcommand{\Herschel}{\project{Herschel}}
\newcommand{\acronym}[1]{{\small{#1}}}
\newcommand{\PACS}{\project{\acronym{PACS}}}
\newcommand{\SPIRE}{\project{\acronym{SPIRE}}}

\newcommand{\unit}[1]{{\mathrm{#1}}}
\newcommand{\mum}{\unit{\mu m}}

\begin{document}

\title{Inferring dust properties from 6-band \Herschel\ imaging \\ 
       but at the resolution of the highest resolution band}
\author{MK, DL, DWH}

\begin{abstract}
In many astronomical projects, images or angular maps are built from
multi-wavelength imaging that is also multi-resolution:
Every wavelength is imaged at a different angular resolution.
One method for uniformizing the resolution is to convolve or blur the
images to a common (usually the lowest) resolution, discarding
information.
Of the alternative approaches, the best from an information-theory
perspective is to forward model the full stack of multi-band,
multi-resolution images, accounting for the unique, finite resolution
of each image.
To make the problem well-posed, the optimization or inference must be
regularized with prior information about what kinds of spectral energy
distributions (\acronym{SED}s) are possible, and about the smoothness of the map.
Here we perform this forward modeling on 6-band \PACS\ and
\SPIRE\ imaging of the Andromeda Galaxy to map the dust at resolution
of the highest resolution band (\PACS~$70\,\mum$).
We regularize the optimization with a simple model of dust emissivity.
We demonstrate that we can produce high-resolution dust density and
temperature maps, and make predictions for (hypothetical future)
higher-resolution infrared imaging.
We demonstrate the validity of the maps with cross-validation-like
tests of their predictive power.
We discuss limitations and extensions of the method, especially as
regards the flexibility of the \acronym{SED} model.
\end{abstract}

\section{Introduction}

Hello World.

\section{Method}

Hello World.

\section{Experiments}

Hello World.

\section{Discussion}

Hello World.

\acknowledgements
Brent Groves, Karin Sandstrom, Tomas Henning, etc.
Grant numbers, etc.
\acronym{NASA ADS}, etc.

\end{document}
